\documentclass[12pt]{article}
\usepackage[breaklinks=true]{hyperref}
\usepackage{color}
\usepackage{amsmath,amssymb,amsthm}
\usepackage{natbib}
\usepackage[margin=0.75in]{geometry}
\usepackage[singlespacing]{setspace}
\usepackage[bottom]{footmisc}
\usepackage{floatrow}
\usepackage{float,graphicx}
\usepackage{enumerate}


\newtheorem{theorem}{Theorem}[section]
\newtheorem{lemma}[theorem]{Lemma}
\newtheorem{assumption}{Assumption}

\newcommand{\beq}{\begin{equation}}
\newcommand{\eeq}{\end{equation}}


\newcommand{\todo}[1]{{\color{red}{TO DO: \sc #1}}}

\newcommand{\reals}{\mathbb{R}}
\newcommand{\integers}{\mathbb{Z}}
\newcommand{\naturals}{\mathbb{N}}
\newcommand{\rationals}{\mathbb{Q}}

\newcommand{\ind}{\mathbb{I}} % Indicator function
\newcommand{\pr}{\mathbb{P}} % Generic probability
\newcommand{\ex}{\mathbb{E}} % Generic expectation
\newcommand{\var}{\textrm{Var}}
\newcommand{\cov}{\textrm{Cov}}

\newcommand{\normal}{N} % for normal distribution 
\newcommand{\eps}{\varepsilon}
\newcommand\independent{\protect\mathpalette{\protect\independenT}{\perp}}
\def\independenT#1#2{\mathrel{\rlap{$#1#2$}\mkern2mu{#1#2}}}
\newcommand{\argmax}{\textrm{argmax}}
\newcommand{\argmin}{\textrm{argmin}}

\title{Outline: PRNGs and Permutations}
\author{Kellie Ottoboni}
\date{Draft \today}
\begin{document}
\maketitle

%\newpage

%\begin{abstract}


%\end{abstract}

%\newpage

\begin{itemize}
\item The first order of business is to investigate how R and Python generate pseudo-random numbers and
what algorithms they use to sample, permute, shuffle, etc.
This may require looking at the raw code since the R documentation doesn't say how \texttt{sample} works.
\item \textbf{Permutation testing with one sample:} \\
We have $N$ observations and we're doing some permutation test with them.
To approximate the null distribution, we want to sample uniformly at random from all $N!$ permutations of the observations.
Is it possible to obtain all of these permutations, or are we constrained by the period of the PRNG?
\begin{itemize}
\item I am totally open to changing notation! This is temporary.
\item Suppose the period of the PRNG is $\mathcal{P}$.
\item Suppose the permutation algorithm takes $K$ operations.
\item If $K \equiv 0 \mod \mathcal{P}$, then the PRNG will start over at some point. If $\mathcal{P}/K < N!$, then we can't reach all possible permutations.
Otherwise, we're in good shape and we will just start to repeat permutations before the PRNG reaches the end of its period.
\item What happens if $\mathcal{P}/K < N!$ but $K$ does not divide $\mathcal{P}$?
\end{itemize}
\item What is the ``best'' way to do permutations to avoid reaching the end of the period?
There are two issues at tension: the period of the PRNG and the computational complexity of the PRNG and shuffling algorithm.
We want to balance computational efficiency with correctness.
\item What happens if we generate pseudo-random numbers in a distributed fashion?
Obviously one has to set the seed differently for each thread, but does this improve the risk of repeating?
\end{itemize}

\bibliographystyle{plainnat}
\bibliography{refs}


\end{document}